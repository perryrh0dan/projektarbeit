% !TEX root = ../Projektdokumentation.tex

% Anpassung an Landessprache ---------------------------------------------------
\usepackage{babel}

% Umlaute ----------------------------------------------------------------------
%   Umlaute/Sonderzeichen wie äüöß direkt im Quelltext verwenden (CodePage).
%   Erlaubt automatische Trennung von Worten mit Umlauten.
% ------------------------------------------------------------------------------
\usepackage[T1]{fontenc}
\usepackage{textcomp} % Euro-Zeichen etc.

% Schrift ----------------------------------------------------------------------
\usepackage{lmodern} % bessere Fonts
\usepackage{relsize} % Schriftgröße relativ festlegen

% Tabellen ---------------------------------------------------------------------
\usepackage[table]{xcolor}
\usepackage{tabularx}
% für lange Tabellen
\usepackage{longtable}
\usepackage{color, array}
\usepackage{ragged2e}
\usepackage{lscape}
\usepackage{tabu}
%\newcolumntype{w}[1]{>{\raggedleft\hspace{0pt}}p{#1}} % Spaltendefinition rechtsbündig mit definierter Breite

% Grafiken ---------------------------------------------------------------------
\usepackage[dvips,final]{graphicx} % Einbinden von JPG-Grafiken ermöglichen
\usepackage{graphics} % keepaspectratio
\usepackage{floatflt} % zum Umfließen von Bildern
\graphicspath{{Bilder/}} % hier liegen die Bilder des Dokuments

% Sonstiges --------------------------------------------------------------------
\usepackage[titles]{tocloft} % Inhaltsverzeichnis DIN 5008 gerecht einrücken
\usepackage{amsmath,amsfonts} % Befehle aus AMSTeX für mathematische Symbole
\usepackage{enumitem} % anpassbare Enumerates/Itemizes
\usepackage{xspace} % sorgt dafür, dass Leerzeichen hinter parameterlosen Makros nicht als Makroendezeichen interpretiert werden

\usepackage{makeidx} % für Index-Ausgabe mit \printindex
\usepackage{acronym} % es werden nur benutzte Definitionen aufgelistet

% Einfache Definition der Zeilenabstände und Seitenränder etc.
\usepackage{setspace}
\usepackage{geometry}

% Symbolverzeichnis
\usepackage[intoc]{nomencl}
\let\abbrev\nomenclature
\renewcommand{\nomname}{Abkürzungsverzeichnis}
\setlength{\nomlabelwidth}{.25\hsize}
\renewcommand{\nomlabel}[1]{#1 \dotfill}
\setlength{\nomitemsep}{-\parsep}

\usepackage{varioref} % Elegantere Verweise. „auf der nächsten Seite“
\usepackage{url} % URL verlinken, lange URLs umbrechen etc.

\usepackage{chngcntr} % fortlaufendes Durchnummerieren der Fußnoten
% \usepackage[perpage]{footmisc} % Alternative: Nummerierung der Fußnoten auf jeder Seite neu

\usepackage{ifthen} % bei der Definition eigener Befehle benötigt
\usepackage{todonotes} % definiert u.a. die Befehle \todo und \listoftodos
\usepackage[square]{natbib} % wichtig für korrekte Zitierweise

\usepackage{etoolbox}
\preto\section{\clearpage} % jede Sektion soll auf einer neuen Seite beginnen

\usepackage[section]{placeins}

% PDF-Optionen -----------------------------------------------------------------
\usepackage{pdfpages}
\pdfminorversion=5 % erlaubt das Einfügen von pdf-Dateien bis Version 1.7, ohne eine Fehlermeldung zu werfen (keine Garantie für fehlerfreies Einbetten!)
\usepackage[
    bookmarks,
    bookmarksnumbered,
    bookmarksopen=true,
    bookmarksopenlevel=1,
    colorlinks=true,
% diese Farbdefinitionen zeichnen Links im PDF farblich aus
    linkcolor=CamelotBlau, % einfache interne Verknüpfungen
    anchorcolor=CamelotBlau,% Ankertext
    citecolor=CamelotBlau, % Verweise auf Literaturverzeichniseinträge im Text
    filecolor=CamelotBlau, % Verknüpfungen, die lokale Dateien öffnen
    menucolor=CamelotBlau, % Acrobat-Menüpunkte
    urlcolor=CamelotBlau,
% diese Farbdefinitionen sollten für den Druck verwendet werden (alles schwarz)
    %linkcolor=black, % einfache interne Verknüpfungen
    %anchorcolor=black, % Ankertext
    %citecolor=black, % Verweise auf Literaturverzeichniseinträge im Text
    %filecolor=black, % Verknüpfungen, die lokale Dateien öffnen
    %menucolor=black, % Acrobat-Menüpunkte
    %urlcolor=black,
%
    %backref, % Quellen werden zurück auf ihre Zitate verlinkt
    pdftex,
    plainpages=false, % zur korrekten Erstellung der Bookmarks
    pdfpagelabels=true, % zur korrekten Erstellung der Bookmarks
    hypertexnames=false, % zur korrekten Erstellung der Bookmarks
    linktocpage % Seitenzahlen anstatt Text im Inhaltsverzeichnis verlinken
]{hyperref}
% Befehle, die Umlaute ausgeben, führen zu Fehlern, wenn sie hyperref als Optionen übergeben werden
\hypersetup{
    pdftitle={\titel -- \untertitel},
    pdfauthor={\autorName},
    pdfcreator={\autorName},
    pdfsubject={\titel -- \untertitel},
    pdfkeywords={\titel -- \untertitel},
}
