% !TEX root = Projektdokumentation.tex
\textbf{COR1-3}:
\linebreak
Diese Transaktion ist zum Erstellen, Bearbeiten und Anzeigen von Prozessaufträgen.

\textbf{/SAPAPO/RRP3}:
\linebreak
Mithilfe der Produktsicht lassen sich zu einer vorher definierten Produkt-Werkskombination alle Forecasts, Prozessaufträge, Planaufträge, Produktionsaufträge, Kaufanforderungen und vieles mehr anzeigen.

\textbf{CMOD}:
\linebreak
Mittels dieser Transaktion können Erweiterungspunkte ausgewählt und implementiert werden.

\textbf{se18}:
\linebreak
Diese Transaktion dient zur Implementierung von \ac{BAdI}s.

\textbf{se80}
\linebreak
Die se80 Transaktion ist die in meinem Projekt am meisten genutzte Transaktion. Mit ihr wird programmiert und es können Datenbanktabellen und Dictionary-Objekte erstellt werden.
