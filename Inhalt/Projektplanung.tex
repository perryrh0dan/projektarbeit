% !TEX root = ../Projektdokumentation.tex
\section{Projektplanung} 
\label{sec:Projektplanung}


\subsection{Projektphasen}
\label{sec:Projektphasen}
Zur Umsetzung des Projekts standen mir 70 Stunden zur Verfügung. Diese waren sowohl für die Umsetzung des Projekts als auch für die Dokumentation. Vor Projektbeginn wurde das Projekt in mehrere Phasen gesplittet und die Stunden aufgeteilt. Eine grobe Übersicht kann der nachfolgenden Tabelle entnommen werden. 

Tabelle~\ref{tab:Zeitplanung} zeigt ein Beispiel für eine grobe Zeitplanung.
\tabelle{Zeitplanung}{tab:Zeitplanung}{ZeitplanungKurz}\\
Eine detailliertere Zeitplanung findet sich im \Anhang{app:Zeitplanung}.


\subsection{Ressourcenplanung}
\label{sec:Ressourcenplanung}
In der Ressourcenübersicht, welche sich im Anhang befindet, sind alle Ressourcen aufgelistet, die für das Projekt eingesetzt wurden. Damit sind sowohl Hardware, Software und das Personal gemeint. Es wurde darauf geachtet, dass nur Software zum Einsatz kommt welche entweder kostenfrei (z.B. als Freeware oder gar Open Source) angeboten werden oder die Camelot AG bereits Lizenzen für diese besitzt und zur Verfügung stellen kann. Dadurch sollen die Projektkosten möglichst geringgehalten werden. 


\subsection{Entwicklungsprozess}
\label{sec:Entwicklungsprozess}
TODO
