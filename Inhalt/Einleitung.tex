% !TEX root = ../Projektdokumentation.tex
\section{Einleitung}
\label{sec:Einleitung}
Das folgende Projekt ist mein IHK-Abschlussprojekt, welches im Rahmen meiner Ausbildung zum Fachinformatiker im Bereich Anwendungsentwicklung durchgeführt wurde.

\subsection{Vorstellung des Betriebs und meiner Selbst} 
\label{sec:Vorstellung des Betriebs und meiner Selbst}
Am 01. März 2017 habe ich meine Ausbildung bei der Spreitzenbarth Consultants GmbH begonnen, einem Unternehmen mit 50 Mitarbeitern und Fokus auf Supply Chain Management. Dort habe ich hauptsächlich Web Anwendungen mit ASP.NET MVC und AngularJs sowie Windows Applikationen mit C\# programmiert. Aufgrund der Firmenauflösung musste ich meinen Ausbildungsbetrieb wechseln und bin seit 01. Juni 2018 bei der Camelot ITLab GmbH beschäftigt. Hier beschäftige ich mich mit dem gesamten SAP Umfeld und lerne die Programmiersprache \ac{ABAP}.

\subsection{Projektziel} 
\label{sec:Projektziel}
Im SAP \ac{APO} bzw. im SAP \ac{ECC} gibt es verschiedene Arten von Aufträgen. Ein Auftrag im allgemeinen umfasst Produkte, Mengen, Ressourcen(Maschinen) und Zeit. Es gibt Planaufträge, eine Art Beschaffungsvorschlag, welche als internes planerisches Element genutzt werden und jederzeit umgebucht werden können. Diese werden dann Richtung Heutelinie zu  Prozess- bzw. Produktionsaufträge konvertiert, welche dann fix gebucht sind und nicht mehr geändert werden können. Der Planer hat die Möglichkeit, Prozess-, Produktions- und Plan-Aufträge manuell anzulegen bzw. vorhandene Planaufträge zu editieren. Allerdings is es nicht Möglich, Notizen oder Beschreibungen für diese zu hinterlegen. Projektziel ist die Erstellung von Programmen, die es dem Planer ermöglichen, Kommentare	für Aufträge zu erstellen und anzusehen. Außerdem sollen bereits vorhandene Programme entsprechend erweitert werden, um den Planer bestmöglich zu unterstützen. Da dieses Projekt ein internes Projekt ist, wurden alle Anforderungen in innerbetrieblichen Besprechungen erarbeitet. Bei der Implementierung wurde besonders darauf geachtet, einen leicht zu erweiternden Code zu produzieren, da in zukünftigen Versionen noch weiter Feature hinzugefügt werden sollen, wie z.B. ein Kommentarverlauf, sodass der Planer auch vorherige Versionen von Kommentaren anschauen kann. 

\subsection{Projektumfeld}
\label{sec:Projektumfeld}
Die Camelot ITLab GmbH ist eine im SAP-Umfeld tätige Unternehmensberatung, die sowohl funktionale als auch technische Implementierungen von Geschäftsprozessen umsetzt.	Die Abteilung SCM Solution Development innerhalb der Camelot ITLab GmbH, in deren Umfeld	auch dieses Projekt umgesetzt wurde, ist für die softwareseitige Implementierung technischer	Anforderungen zuständig. Der Schwerpunkt der Abteilung ist Supply Chain Management, im Speziellen Produktions- und Feinplanung.

\subsection{Projektbeteiligte Personen} 
\label{sec:Projektbeteiligte Personen}
Tabelle~\ref{tab:BeteiligtePer} zeigt alle Personen, die an dem Projekt beteiligt waren.
\tabelle{Beteiligte Personen}{tab:BeteiligtePer}{BeteiligtePersonen}

\subsection{Projektbegründung} 
\label{sec:Projektbegruendung}
Da die im ~\nameref{sec:Projektziel} beschriebenen Funktionalitäten für den Planer äußerst hilfreich sind und diese schon oft von Kunden angefragt wurden, da es etwas Vergleichbares auf dem Markt bzw. im SAP Standard noch nicht gibt, wurde dieses Projekt von der Camelot ITLab GmbH angenommen. 