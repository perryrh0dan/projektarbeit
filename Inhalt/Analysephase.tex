% !TEX root = ../Projektdokumentation.tex
\section{Analysephase} 
\label{sec:Analysephase}

\subsection{Ist-Analyse} 
\label{sec:IstAnalyse}
Das Projekt wurde im Umfeld der Produktionsplanung durchgeführt und bezieht sich auf Plan-, Prozess- und Produktionsaufträge. Der Planer hat im SAP \ac{APO} die	Möglichkeit, Zugangs- und Abgangselemente manuell anzulegen. In Folgeprozessen kann es zu Problemen kommen, wenn die Gründe dieser Planänderungen nicht sichtbar sind. Bisher müssen solche Änderungen ohne Systemunterstützung per Email oder auf anderem Weg allen Beteiligten mitgeteilt werden.

\subsection{Wirtschaftlichkeitsanalyse}
\label{sec:Wirtschaftlichkeitsanalyse}
Aufgrund der Probleme, die bereits in der ~\nameref{sec:Projektbegruendung} und der ~\nameref{sec:IstAnalyse} beschrieben wurden, ist ersichtlich, wie wichtig und hilfreich dieses Projekt ist. Trotzdem wurde im folgenden Abschnitt analysiert, ob die Umsetzung auch aus wirtschaftlichen Gesichtspunkten gerechtfertigt ist.

\subsubsection{Projektkosten}
\label{sec:Projektkosten}
Es wurden neben den Kosten, die für Hardware und Software anfallen, auch die Personalkosten berücksichtigt. 
\newline
Da die genauen Personalkosten Betriebsgeheimnis sind, wurde die Kalkulationen mit groben Stundensätzen durchgeführt. Der Stundensatz eines Auszubildenden ergab sich aus dem Monatsgehalt von ca. 1000€.
		
\begin{eqnarray}
	365 \mbox{ Tage/Jahr} \cdot \frac{5}{7} - 30 Tage = 230 \mbox{ Tage/Jahr}\\
	8 \mbox{ h/Tag} \cdot 230 \mbox{ Tage/Jahr} = 1840 \mbox{ h/Jahr}\\
	\eur{1000}\mbox{/Monat} \cdot 12 \mbox{ Monate/Jahr} = \eur{12000} \mbox{/Jahr}\\
	\frac{\eur{12000} \mbox{/Jahr}}{1840 \mbox{ h/Jahr}} \approx \eur{6,52}\mbox{/h}
\end{eqnarray}
	
Der Rechnung zufolge beträgt der Stundensatz eines Azubis ca. 7€. Die Kosten eines Mitarbeiters wurden auf 25€ die Stunde geschätzt. 
\newline
Um die anfallenden Hardwarekosten zu berechnen, wurden die Anschaffungskosten aller Geräte summiert und dann durch die durchschnittliche Lebenszeit in Arbeitsstunden, in diesem Fall 3 * 365 * 5/7 * 8, geteilt.
	
Tabelle~\ref{tab:Hardwarekosten} zeigt die Hardwarekosten, die für die Geräte eines Mitarbeiters anfallen.
\tabelle{Hardwarekosten}{tab:Hardwarekosten}{Hardwarekosten}\\

\begin{eqnarray}
	\frac{\eur{1680}}{3\mbox{ Jahre}} = 560 \mbox{/Jahr}\\
	\frac{\eur{560} \mbox{/Jahre}}{1840 \mbox{ h/Jahr}} \approx \eur{0.30}\mbox{/h}
\end{eqnarray}

Damit ergab sich ein Stundensatz von \eur{0.3} für die Hardware.
\newline
Sonstige wesentliche Kosten neben den Hardwarekosten sind natürlich auch Arbeitsplatz, Möblierung, Strom, Internet und Lizenzen, welche einen Großteil der Ressourcenkosten ausmachen. Durch grobe Hochrechnung dieser ergaben sich insgesamt Ressourcenkosten (ohne Personal) von ca. \eur{10} als Stundensatz.
\tabelle{Projektkosten}{tab:Projektkosten}{Projektkosten.tex}\\
Wie in Tabelle~\ref{tab:Projektkosten} ermittelt, betrugen die gesamten Projektkosten \eur{1510}.

\subsubsection{Amortisationsdauer}
\label{sec:Amortisationsdauer}
Im folgenden Abschnitt wird gezeigt, wann sich die Entwicklung aus wirtschaftlicher Sicht für die Camelot ITLab GmbH und für den Kunden lohnt. Die Amortisationsdauer wird berechnet, indem man die Produktionskosten bzw. Anschaffungskosten durch die Gewinne bzw. Kostenersparnis, die aufgrund dieses Produktes entstanden sind, dividiert.

\paragraph{Camelot ITLab GmbH}\mbox{} \\
Für dieses Tool wird ein Preis von etwa 500€ angesetzt. Der Break-Even-Point (Gewinnschwelle) liegt also bei
\begin{eqnarray}
\frac{\eur{1510}}{\eur{500}} \approx 3 \mbox{ verkauften Einheiten} 
\end{eqnarray}

\paragraph{Kunde}\mbox{} \\
Hier lassen sich die Gewinne nicht so leicht erfassen, da zum einen die Fehleranfälligkeit der alten Lösungen (Kommentare per Email verschicken, Zettelwirtschaft usw.) deutlich verringert werden, zum anderen erspart das Tool dem Planer, welcher die Aufträge anlegt, deutlich Zeit, da dieser nicht noch ein weiteres Programm benötigt, um die Kommentare einzutragen und an seine Kollegen zu verteilen. Des Weiteren ermöglicht es den Mitarbeitern, an der Ressource (Maschine) Kommentare direkt in der Transaktion neben den Aufträgen zu sehen. In der folgenden Beispielrechnung wurden drei Szenarien einmal durchgerechnet. 

\paragraph{Beispielrechnung}
Es wird davon ausgegangen, dass ein Planer jeden Tag ca. eine Stunde damit beschäftigt ist, Aufträge zu kommentieren. Dieses Tool bietet eine Zeitersparnis von ca. 50\%, da die Kommentare direkt in der Transaktion verfasst werden können. Dies bedeutet eine Zeiteinsparung von 30 Minuten pro Tag und pro Planer. Bei etwa 230 Arbeitstagen pro Jahr ergibt sich eine gesamte Zeiteinsparung von  
\begin{eqnarray}
230 \mbox{ Tage/Jahr} \cdot 30 \mbox{ min/Tag} = 6900 \mbox{ min/Jahr} = 115 \mbox{ h/Jahr} 
\end{eqnarray}

Das Gehalt eines Produktionsplaners beträgt etwa \eur{14} die Stunde. Die tatsächlichen Kosten, incl. Nebenkosten, die für die Firma anfallen, werden auf ca. 20€ geschätzt. Dadurch ergibt sich eine jährliche Einsparung von 
\begin{eqnarray}
115 \mbox{h} \cdot \eur{(20)}{\mbox{/h}} = \eur{2300}
\end{eqnarray}

Je nach Firmengröße und Anzahl der Planer variiert die Amortisationszeit stark. Daher wurden Anhand dreier Beispiele die Zeit berechnet.

Die Amortisationszeit für eine kleine Firma mit 2 Planern beträgt:
$\frac{\eur{500}}{2 \cdot \eur{2300€} \mbox{/Jahr} } \approx 0,10 \mbox{ Jahre} \approx 5 \mbox{ Wochen}$.

Die Amortisationszeit für eine mittelgroße Firma mit 50 Planern beträgt
$\frac{\eur{500}}{50 \cdot \eur{2300€} \mbox{/Jahr} } \approx 0,004 \mbox{ Jahre} \approx 1,5 \mbox{ Tage}$.

Die Amortisationszeit für eine große Firma mit 200 Planern beträgt
$\frac{\eur{500}}{200 \cdot \eur{2300€} \mbox{/Jahr} } \approx 0,001 \mbox{ Jahre} \approx 9 \mbox{ Stunden}$.

Man erkennt, dass das Projekt sowohl für die Camelot ITLab GmbH als auch für den Kunden wirtschaftlich Sinnvoll ist.

\subsection{Nutzwertanalyse}
\label{sec:Nutzwertanalyse}
In der folgenden Tabelle ~\ref{tab:Nutzwertanalyse} werden noch einige weitere Punkte aufgeführt, welche abseits der primären finanziellen Aspekte für diese Produkt bzw. die alten Verfahren sprechen. Für Kriterien, die in direktem Zusammenhang mit der Produktion und den erzielten Ergebnissen stehen, wird eine hohe Gewichtung angesetzt. Kriterien wie etwa Komfort, welche abhängig von persönlichen Präferenzen des Planers sind, werden eher niedrig eingestuft.

\tabelle{Nutzwertanalyse}{tab:Nutzwertanalyse}{Nutzwertanalyse}

Sowohl die Wirtschaftlichkeitsanalyse als auch die Nutzwertanalyse haben gezeigt wie sinnvoll und hilfreich diese Projekt ist.

\subsection{Anwendungsfälle}
\label{sec:Anwendungsfaelle}
Während der Analysephase wurden einige der typischen Anwendungsfälle, die von den umzusetzenden Programmen und Erweiterungen abgedeckt werden sollen, mittels eines Use-Case-Diagramms zusammengetragen, um einen groben Überblick über diese zu erhalten. Dieses Diagramm befindet sich im \Anhang{app:UseCase}.

\subsection{Qualitätsanforderungen}
\label{sec:Qualitaetsanforderungen}
Wie bei jedem anderen Projekt der Camelot ITLab GmbH gelten auch hier die Programmierrichtlinien des Unternehmens. Eine Auszug mit den für dieses Projekt wichtigen Punkten findet sich im \Anhang{app:Programmierrichtlinien}. Außerdem muss es für den Administration Screen eine Eingabeüberprüfung geben, die verhindert, dass der User Keyfelder bzw. Felder, die in der Datenbanktabelle nicht vorhanden sind, angibt.

\subsection{Lastenheft/Fachkonzept}
\label{sec:Lastenheft}
Das Lastenheft beschreibt die vom ``Auftraggeber festgelegte Gesamtheit der Forderungen an die Lieferungen und Leistungen eines Auftragnehmers innerhalb eines Auftrags``(\cite{Wiki.Induux}). Das Lastenheft, welches im Bezug auf dieses Projekt entstanden ist, befindet sich im \Anhang{app:Lastenheft}. 

\Zwischenstand{Analysephase}{Analyse}