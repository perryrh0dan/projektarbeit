\subsection{Programmierrichtlinien}
\label{app:Programmierrichtlinien}
Im folgenden Abschnitt werden kurz die am häufigsten gebrauchten Programmierrichtlinien des CAMELOT ITLabs aufgeführt.

\subsubsection{Bennenung}
\label{Bennenung}
\paragraph{Variablen}
Globale Variablen z.B. Variablen in Reports, müssen immer mit einem g(global) anfangen. 
Lokale Variablen z.B. in Methoden von Klassen, müssen immer mit einem l(local) anfangen.

Darauf folgt dann entweder ein:
\begin{enumerate}
	\item t für Tabellen
	\item s für Strukturen
	\item v für Variablen
\end{enumerate}
	
Darauf folgt dann ein \_ und ein passender Name, welcher die Variable möglichst treffsicher beschreibt.

\paragraph{Klassen:}
Klassen Bezeichnungen starten immer mit dem Paketnamen in welchem sich die Klassen befindet + CL(Class) z.B. /CAMELOT/CL\_OC. Der Paketname ist /CAMELOT/OC und in die Mitte wird ein CL gehängt.

\subsubsection{Formatierung}
\label{Formatierung}
\paragraph{Pretty-Printer:}
Der Pretty Printer, ein Tool welches den Quellcode aufbereitet und somit dem Entwickler eine bessere Lesbarkeit bietet muss ,wie in der folgenden Grafik dargestellt, eingestellt werden. Der Pretty-Printer ist über die Schaltfläche [Pretty Printer] oder die Tastenkombination "Umschalt + F1" aufzurufen.
\begin{figure}[htb]
	\centering
	\includegraphicsKeepAspectRatio{PrettyPrinterSettings.png}{0.6}
	\caption{Einstellungen des Pretty-Printers}
\end{figure}



	

