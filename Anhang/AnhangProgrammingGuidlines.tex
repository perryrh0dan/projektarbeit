\subsection{programmierrichtlinien}
\label{app:programmierrichtlinien}
Im folgenden Abschnitt werden kurz die am häufigsten gebrauchten programmierrichtlinien des CAMELOT ITLabs aufgeführt.

\subsubsection{Bennenung}:
\paragraph{Variablen}
Globale Variablen z.B. Variablen in Reports, müssen immer mit einem g(global) anfangen. 
Lokale Variablen z.B. in Methoden von Klassen, müssen immer mit einem l(local) anfangen.

Darauf folgt dann entweder ein:
	t für Tabellen
	s für Strukturen
	v für Variablen
	
Darauf folgt dann ein \_ und ein passender Name welcher die Variable möglichst treffsicher beschreibt.

\paragraph{Klassen}
Klassen Bezeichnungen starten immer mit dem Paketnamen in welchem sich die Klassen befindet + CL(Class) z.B. /CAMELOT/CL\_OC. Der Paketname ist /CAMELOT/OC und in die Mitte wird ein CL gehängt.
	

