\subsection{Pflichtenheft (Auszug)}
\label{app:Pflichtenheft}

\subsubsection*{Zielbestimmung}

\begin{enumerate}[itemsep=0em,partopsep=0em,parsep=0em,topsep=0em]
\item Musskriterien % Wikipedia: für das Produkt unabdingbare Leistungen, die in jedem Fall erfüllt werden müssen
	\begin{enumerate}
	\item ECC System
		\begin{itemize}
		\item Ein neues Package muss im ECC angelegt werden mit dem Namen \textsc{/camelot/oc}.
		\item Die AUFK Datenbanktabelle muss um ein Feld mit dem Namen \textsc{zz\_order\_com-ment} erweitert werden.
		\item Ein extra Programm, welches eine gefilterte Auswahl an Prozess-, Plan- und Produktions-Aufträgen anzeigt, muss erstellt werden.
		\item Ebenfalls sollen hier jeweils die Kommentare für die Aufträge angezeigt und vom User geändert werden können. 
		\item Weiterhin soll dieses Programm massenänderungsfähig sein. 
		\item Außerdem soll es ein zweites Programm geben, in die der User ein Tabellenfeld der AUFK Datenbanktabelle eingeben kann, welches dann für die Kommentare genutzt wird.
		\item Mittels Input Checks soll verhindert werden, dass der User ein Feld eingibt, das es in der AUFK Tabelle nicht gibt.
		\item Zum Schluss soll die COR1-3 erweitert werden.
		\begin{itemize}
			\item Ein neues Package muss angelegt werden mit dem Namen ZPP, in welches dann die COR Erweiterung implementiert wird.
			\item Mittels der cmod Transaktion muss ein neues Projekt mit dem Namen \textsc{z\_cor} angelegt werden und zwei Erweiterungspunkte hinzugefügt werden (PPCO0007 und PPCO0020).
		\end{itemize}
		\end{itemize} 
	\item APO System 
		\begin{itemize}
		\item Es muss ebenfalls ein neues Package angelegt werden \textsc{/camelot/oc}
		\item Die \textsc{/sapapo/ordflds} muss um ein Feld mit dem Namen \textsc{order\_comment} erweitert werden.
		\item Ebenso wie im ECC System soll es ein Programm geben, welches mittels vom User eingegebener Order Number einen Auftrag mit Kommentar anzeigt, der vom User geändert werden kann.
		\item Des Weiteren soll die RRP3 erweitert werden.
		\begin{itemize}
			\item Es muss ein neues Package mit dem Namen \textsc{zscm} angelegt werden.
			\item Mittels der se18 muss ein BAdI in dieses Package implementiert werden.
			\item Die Logik muss programmiert werden.
			\item Eine implizite Erweiterung muss angelegt werden, um die t\_style Tabelle zu manipulieren.
		\end{itemize}
		\end{itemize}
	\end{enumerate}
\end{enumerate}
